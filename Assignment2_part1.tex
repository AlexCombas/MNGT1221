\documentclass[12pt,titlepage]{article}

\usepackage[margin=1.5in,paperwidth=8.5in, paperheight=11in]{geometry}
\usepackage[T1]{fontenc}
\usepackage[utf8]{inputenc}
\usepackage[english]{babel}
\usepackage{microtype}
\usepackage{fancyhdr}
\usepackage{amsfonts}
\usepackage{amsmath}

\pagestyle{fancy}
\fancyhf{}
\lhead{Assignment 2}
\cfoot{\thepage}
\renewcommand{\familydefault}{\sfdefault}

\begin{document}

\title{\Huge{\textbf{Assignment 2}}}
\author{By Alexandros Combas \#T00655895}
\date{May 10, 2020}
\maketitle


\renewcommand\thepart{\Alph{part}}
\part{Problem Solving and Decision-Making (35 Marks)}
\rhead{Part A}

\section{What was the difference between the way Scott and Perez viewed the problem in this case? (10 marks)}
	
Scott, the general manager, looked at the problem from a top-down perspective. He thought that the fundamental problem with his sales staff was that they were too slow to meet and greet customers, and to determine the buyers' needs. If this was truly the basic problem then the solution would be for the staff to become quicker at doing their job. However, the way that Perez earned his many successes was not through speed, in fact just the opposite. Instead of rushing his customer he would allow them to casually talk with other members of the staff, and then the other staff would pass Perez pertinent tips about what the customers truly wanted, as well as other information that the staff had picked up during their casual conversation with the customer.

There was also another difference, Scott believed that good sales
numbers were the result of good individual salespeople, however, Perez demonstrated that it took a close-knit team of friends working together to generate his own high sales numbers.

\section{How did the initial problem definition affect the approach to, and solution of, the problem? (10 marks)}

The initial way that Scott looked at the problem was that the individual salespeople needed to simply work harder and faster to improve their sales, this lead him to believe that the solution to this problem could be discovered by a brainstorming session with other members of management. Presumably if he had continued with this line of thinking he would have then written some tips to be handed out to the sales force that were based on the results from the brainstorming meeting. Although it is possible that these tips would have been helpful, it is doubtful that this would have resulted in a dramatic turn around in sales numbers.

The actual solution to the problem was found among the salespeople themselves, as the successful salesman Perez was promoted to a trainer/manager position, and then he personally worked together with the other members of the sales force to share his proven techniques of becoming friends with the other members of the company and passing along to them gifts and treats when they gave him successful tips. Perez working closely
with the other members of the sales team also generated brand-new ideas for how to improve sales all on its own.

\section{List the major way in which Scott can apply what he learned from this situation to continue improving the quality and productivity of the new car sales force? (10 marks)}

Going forward it would make sense that Scott would continue to look for the highly successful salespeople who have proven themselves over a number of years and then promote them to positions where they can pass their training and skills onto the other workers. It would also be a good idea for Scott to try to talk more often with members of his sales staff, if he had not stopped to speak with Perez then none of this would have happened. Finally, since Perez's technique relies heavily upon giving gifts and treats to other members of the company who would supply him with important tips about customers it would
be a good idea to formalize this system and give those members of
the company who contribute to a sale a monetary bonus as an incentive.

\section{Does the decision-making process vary depending on where a manager or supervisor is located in the managerial hierarchy? Discuss. (5 marks)}

Higher-level managers will tend to look at the company from a different perspective than a lower-level manager. Upper management concern themselves with the bigger picture and often fail to notice the smaller details that a manager who is closer to the front line will pick up on. For example, Scott tried to come up with some big picture solution to the problems by having a brainstorming session with other managers, but Perez was successful because he was willing to buy lunch for the cashiers or others if they gave him a good tip, and that is a simple
and low-level concept that upper managers probably wouldn't come up with on their own.


\part{Organizing (35 marks)}
\setcounter{section}{0}
\rhead{Part B}

\section{Using the concept of unity of command, explain the cause of Simpson's angry outburst. (5 marks)}

The concept of "unity of command" states that each employee should only have one supervisor to whom he or she is directly responsible. The reason for Simspon's angry outburst was because this concept was violated when she was assigned to the "Enter Japan" team and given a second set of duties and a second supervisor which conflicted with the time that she needed to perform her original duties for her original supervisor. Assigning Simpson to two jobs in this manner created a no-win situation for her which caused her to become increasingly frustrated and stressed as she was not able to perform either of her jobs at appropriate levels. This was even worse in Simpson's case since she was known to be a consistent high performer and could now see her previously good performance record being destroyed. 

\section{What could White have done differently when he first was asked to appoint someone from his department to the team? What could he have done to ensure that Simpson's workload was more manageable? (15 marks)}

There are a number of things that White could have done differently. First, he should have requested a detailed job description since without an accurate description it is impossible to know how much work a job requires. The description he was given at the beginning was that the employee would be required to research shipping issues involved in getting the company's kits to Japan, but this is too general to be considered an accurate job description. 

When White asked Simpson to take on this additional project she expressed to him that she was not sure she was the right person for the job since it was different from anything she had ever done before. White should have listened to her and taken her warning into consideration. The fundamental problem here is that although Simpson was good at her existing job she was an unknown quantity when it came to the new job. It was a mistake for him to assume that since she was good at the existing job that she would be equally good when she was assigned to a completely different job in which she had no previous experience. Instead of simply picking his best worker from the existing job, White should have tried to find the best worker for the new job. One way this could have been accomplished would be to send the detailed job description to each of the twelve employees that he supervises and asked if any of them have had previous experience with something similar. If no one from his team had any experience, then he could have at the very least asked if anyone thought that the new job sounded fun and interesting and would like to volunteer, that way he could have at least found someone with a positive attitude. 

Having two supervisors is not in itself inherently wrong since there are cases where that can be done appropriately, however, it does violate the unity of command and so it is something that must be managed with care if it is to be done successfully. White could have improved the situation by requiring weekly updates from Simpson on exactly what she was being asked to do and how long it was taking her to do it. This would have allowed him to recognize that the situation was becoming toxic before it reached dangerous levels. He then could have either scaled back his requirements on Simpson in his department, or he could have requested that the other manager lighten up her own work requirements for Simpson. 

\section{What should Simms do now, both to fill the vacant position and to prevent a similar problem from occurring in the future? (15 marks)}

I believe that the best alternative for Simms if he truly wishes to honor the company policy of promoting from within is to delegate the responsibility for finding a new supervisor to his existing four shift supervisors. They should each be expected to put forth a candidate for recommendation, and then after evaluating the suggested candidates Simms can choose the one he believes is most qualified and suitable for the position. 

The risk here is that this could be seen as abdication rather than delegation, to address this Simms will also need to require that each supervisor take accountability for their decision so that if their suggested nominee is chosen they would then be responsible to mentor the new supervisor and ensure they do an acceptable job and not cause problems with other employees. This way if it turns out that one of his supervisors suggests that Margo James be advanced and it turns out that Margo is the only viable candidate, that supervisor will take responsibility for suggesting her and then through mentoring her hopefully the problem of her being a "control freak" can be fixed before it causes difficulties.  

To prevent this from happening in the future each supervisor should be required to take on a protégé and to give them regular instruction so that if there is ever another situation where a supervisor suddenly needs to be replaced there will already be at least four possible candidates to select from. The proteges could also be periodically rotated to different supervisors, and new employees could be allowed to try if they show interest in studying to become a supervisor. 

If Simms feels that a certain employee would be an excellent candidate to become a future supervisor then he could select that employee to be a protégé of an existing supervisor, and ultimately the decision of whom to choose for the position would always rest with him. 

\part{Staffing and Recruiting (30 marks)}
\setcounter{section}{0}
\rhead{Part C}

\section{As a department supervisor, what decision(s) should Destiney make in regard to this romance? Consider how Destiney should respond to other employees' uneasiness, while still maintaining an ethical stance towards the situation. (15 marks)}

It is important for Destiney to first establish both to the romantic couple and to the other employees the definition of sexual harassment. The definition states that any activity of a sexual nature that creates an environment at work that affects other employees' ability to work, or is considered offensive by other employees is sexual harassment and could be grounds for consequences. 

Destiney also needs to inform Rebecca and Timothy that it is inappropriate for any supervisor to have a romantic relationship with someone that they supervise. 

To maintain her own ethical behavior Destiney should be careful not to expose the existence of their relationship to others in the company. Even though a number of people already know, by exposing this it could be considered humiliating and upsetting to the couple. This can be done by making sure not to refer to them by name in any statements that are made when reminding the employees about the policy regarding workplace romance and sexual harassment. 

At this time however, it seems that none of the employees have filed official complaints regarding this matter so a certain level of lenience towards the company can be justified. Therefore rather than dismissing the romantic couple Destiney should move Rebecca to different departments so that she is no longer directly supervised by Timothy, and also warn them both that any future displays of their relationship at work that make other employees or customers uncomfortable could be grounds for dismissal.

If in the future this romantic relationship becomes a negative situation at the company then having already clearly outlined the policy and given a previous warning Destiney would be well within her rights as department supervisor to dismiss the couple. 

\section{Identify and discuss several areas in which Max Brown's initial experience in Store \#21 could have been improved by a proper orientation. (15 marks)}

The goal of orientation is to ease the new employee into the company by clarifying duties and rectifying misunderstandings, while also explaining how their performance will be evaluated and introducing them to their work environment and co-workers. 

The fact that none of this was done created a very negative impression for Max Brown, and could cause him to quickly resign. 

Rather than sending Max to find an employee, the supervisor should have gone with Max and introduced the new employee to the existing staff. If that had been done then it would have saved time when it was discovered that Kelly was not where the supervisor thought he was. Once Kelly had been found the supervisor could have instructed Kelly to answer any questions Max had and to show him around. Kelly could then have shown Max where to find an apron and how to punch the time clock. 

Kelly could have explained that at this particular store they do not worry so much about the official dress code and let Max know that it would be okay if he wanted to dress more informally from now on as long as he keeps it within certain limits. 

Kelly could have also communicated that the most important task that they need to do is to make sure the shelves are stocked and to work quickly, which would have saved Max from getting in trouble with the supervisor on his very first day. 

\end{document}
