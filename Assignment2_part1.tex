\documentclass[letterpaper,12pt,titlepage]{article}
\usepackage[T1]{fontenc}
\usepackage[utf8]{inputenc}
\usepackage[english]{babel}
\usepackage{microtype}

\renewcommand{\familydefault}{\sfdefault}

\title{\Large{\textbf{Assignment 2}}}
\author{By Alexandros Combas}
\date{May 10, 2020}

\begin{document}
\maketitle

\renewcommand\thepart{\Alph{part}}
\part{Problem Solving and Decision-Making (35 Marks)}

\section{What was the difference between the way Scott and Perez viewed the problem in this case? (10 marks)}
	
The general manager Scott looked at the problem from a top-down perspective, he thought that the fundamental problem with his sales staff was that they were too slow to meet and greet customers, and to determine the buyers' needs. If this was truly the basic problem then the solution would be for the staff to become quicker at doing their job. However, the way that Perez earned his many successes was not through speed, in fact just the opposite. Instead of rushing his customer he would allow them to casually talk with other members of the staff, and then the other staff would pass Perez pertinent tips about what the customers truly wanted, as well as other information that the staff had picked up during their casual conversation with the customer.

There was also another difference, Scott believed that good sales
numbers were the result of good individual salespeople, however, Perez demonstrated that it took a close-knit team of friends working together to generate his own high sales numbers.

\section{How did the initial problem definition affect the approach to, and solution of, the problem? (10 marks)}

The initial way that Scott looked at the problem was that the individual salespeople needed to simply work harder and faster to improve their sales, this lead him to believe that the solution to this problem could be discovered by a brainstorming session with other members of management. Presumably if he had continued with this line of thinking he would have then written some tips to be handed out to the sales force that were based on the results from the brainstorming meeting. Although it is possible that these tips would have been helpful, it is doubtful that this would have resulted in a dramatic turn around in sales numbers.

The actual solution to the problem was found among the salespeople themselves, as the successful salesman Perez was promoted to a trainer/manager position, and then he personally worked together with the other members of the sales force to share his proven techniques of becoming friends with the other members of the company and passing along to them gifts and treats when they have him successful tips. Perez working closely
with the other members of the sales team also generated new ideas
for how to improve sales all on its own.

\section{List the major way in which Scott can apply what he learned from this situation to continue improving the quality and productivity of the new car sales force? (10 marks)}

Going forward it would make sense that Scott would continue to look for the highly successful salespeople who have proven themselves over a number of years and then promote them to positions where they can pass their training and skills onto the other workers. It would also be a good idea for Scott to try to talk more often with members of his sales staff, if he had not stopped to speak with Perez then none of this would have happened. Finally, since Perez's technique relies heavily upon giving gifts and treats to other members of the company who would supply him with important tips about customers it would
be a good idea to formalize this system and give those members of
the company who contribute to a sale a monetary bonus as an incentive.

\section{Does the decision-making process vary depending on where a manager or supervisor is located in the managerial hierarchy? Discuss. (5 marks)}

Higher-level managers will tend to look at the company from a different perspective than a lower-level manager. Upper management concern themselves with the bigger picture and often fail to notice the smaller details that a manager who is closer to the front line will pick up on. For example, Scott tried to come up with some big picture solution to the problems by having a brainstorming session with other managers, but Perez was successful because he was willing to buy lunch for the cashiers or others if they gave him a good tip, and that is a simple
and low-level concept that upper managers probably wouldn't come up with on their own.

\part{Organizing (35 marks)}

\setcounter{section}{0}

\section{Using the concept of unity of command, explain the cause of Simpson's angry outburst. (5 marks)}

This is an test. 

\section{What could White have done differently when he first was asked to appoint someone from his department to the team? What could he have done to ensure that Simpson's workload was more manageable? (15 marks)}

Feel tree to leave. 

\section{What should Simms do now, both to fill the vacant position and to prevent a similar problem from occurring in the future? (15 marks)}

\part{Staffing and Recruiting (30 marks)}

\setcounter{section}{0}

\section{As a department supervisor, what decision(s) should Destiny make in regard to this romance? Consider how Destiney should respond to other employees' uneasiness, while still maintaining an ethical stance towards the situation. (15 marks)}

This is an test. 

\section{Identify and discuss several areas in which Max Brown's initial experience in Store \#21 could have been improved by a proper orientation. (15 marks)}

Feel tree to leave.
\end{document}
